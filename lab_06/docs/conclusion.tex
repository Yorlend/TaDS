\Conclusion

Решение о выборе той или иной структуры данных для произведения операций над данными сильно зависит от поставленных задач. В рамках данной лабораторной работы была выполнена операция удаления элемента из файла, ДДП, АВЛ-дерева и хеш-таблицы.

Самым эффективным по памяти оказался файл, однако его использование выгодно лишь в том случае, когда время выполнения операции не играет такой важной роли, как объем памяти.

Использование деревьев: двоичного дерева поиска и АВЛ-дерева, обеспечивает баланс между временем произведения операций и объемом хранимых данных. В случае с АВЛ-деревом наиболее заметен прирост производительности, но необходимо сохранять сбалансированность дерева для получения столь эффективного по времени результата.

Самой эффективной по времени структурой данных оказалась хеш-таблица. При подходящем выборе максимального размера хеш таблицы и хеш-функции, можно получить среднее количество сравнений немногим превосходящее 1, а в лучшем случае равным 1. Но для хранения хеш-таблицы потребуется в 3-4 раза больше памяти, чем для хранения деревьев.