\Introduction


Большинство  современных  информационных  систем  содержат  в  себе 
базы данных объем которых может превышать несколько гигабайт и к 
поиску  данных  в  таких  базах  предъявляются  особые  требования,  такие  как 
максимальная скорость, прогнозируемость времени поиска и точность 
нахождения информации.

Простые алгоритмы перебора не способны 
обеспечить максимальную скорость и предоставить оценку времени 
выполнения операции поиска ввиду чего повсеместно заменяются на 
алгоритмы  поиска  с  использованием  двоичных  деревьев.  Доказательством 
этого  служит  сравнение  скорости  поиска  в  современных  СУБД.  Именно 
СУБД,  использующие  индексацию  и  двоичные  деревья  поиска  показывают 
наибольшую производительность при поиске данных по таблице, 
содержащей большой объем данных. \cite{avl}

\textbf{Целью данной работы является} получение навыков применения двоичных деревьев, реализация основных операций над деревьями; построение и обработка хеш-таблицы; сравнение эффективности сбалансированных деревьев, двоичных деревьев поиска и хеш-таблиц.

\textbf{Для выполнения поставленной цели необходимо решить следующие задачи: }

\begin{itemize}[$\bullet$]
    \item исследовать структуры данных: деревья и хеш-таблицы;
    \item сформулировать условие задачи;
    \item описать используемые структуры данных;
    \item привести схемы алгоритма балансировки и описание реализуемых хеш-функций;
    \item сформулировать требования к разрабатываемому программному обеспечению;
    \item определить средства программной реализации;
    \item реализовать алгоритм, решающий поставленную задачу;
    \item сравнить эффективность реализованного алгоритма для различных структур данных: сбалансированного дерева, двоичного дерева поиска, хеш-таблицы и файла.
\end{itemize}
