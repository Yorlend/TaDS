\Conclusion

В ходе выполнения лабораторной работы было выявлено, что использование таблицы ключей при сортировке таблицы делает процесс сортировки более эффективным по времени, но также менее эффективным по памяти. Это связано с тем, что нет необходимости работать с массивной структурой, а достаточно просто отсортировать таблицу, состоящую из двух полей. Однако в таком случае затрачивается дополнительная память на хранение этой таблицы.

Стоит отметить, что использование таблицы ключей неэффективно при небольших размерах исходной таблицы. Разница во времени сортировки по ключам и без них несущественна -- в таком случае лучше использовать сортировку самой таблицы, сократив при этом объемы потребляемой памяти.