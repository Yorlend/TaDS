\Conclusion

В результате выполнения работы был реализован алгоритм поиска кратчайшего пути в графе. Были проанализированны графовые структуры, разработаны схемы алгоритмов и описаны используемые структуры данных. Также было произведено тестирование ПО.

Были сделаны следующие выводы:

Хранение графа в виде списков смежности подходит для задач, в которых средняя степень вершины не будет зависеть от числа вершин. Или, другими словами, число ребер будет намного меньше, чем половина от квадрата числа вершин. В случае с поставленной задачей, выгоднее хранить граф в виде матрицы смежности, так как число ребер в большинстве случаев довольно велико.

Выбирать матрицу смежности для хранения графа оправдано в тех случаях, когда число ребер будет близко к числу ребер в полном графе.

Выбор конкретного алгоритма обработки графа зависит от его структуры. При большом числе вершин, но малом числе ребер следует выбирать алгоритмы с временной сложностью $O(N\times M)$ (алгоритм Беллмана - Форда), $O(N\times log(N + M))$ (алгоритм Дейкстры с использованием доп. структур). А если же обрабатываемый граф содержит ребра практически между всеми парами вершин, то будут хорошо работать алгоритмы со сложностью $O(N^2)$ (алгоритм Дейкстры со списком), $O(N^3)$ (алгоритм Флойда-Уоршелла) и $O(N\times M + N^2*logN)$ (алгоритм Джонсона).

В жизни реализованная задача может применяться для проектировании городской инфраструктуры при планировании застройки города администрацией. Информация о самых используемых, густонаселенных или востребованных мест служит отправной точкой для решения проблемы застройки.