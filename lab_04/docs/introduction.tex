\Introduction

Стек --- одна из наиболее часто используемых структур данных в программировании. Данная СД является незаменимой в современном программировании. Стеки используются для организации вызова подпрограмм, адресов возврата.

\textbf{Целью данной лабораторной работы является} реализация операций работы со стеком, который представлен в  виде массива (статического или динамического) и в виде односвязного списка, оценить преимущества и недостатки каждой реализации, получить представление о механизмах  выделения  и  освобождения  памяти  при работе  с  динамическими структурами данных.

\textbf{Для достижения поставленной цели необходимо выполнить следующие задачи:}

\begin{enumerate}
	\item Исследовать варианты реализации стеков.
	\item Привести схемы алгоритмов вставки элемента в стек и удаления из стека.
	\item Описать используемые структуры данных.
	\item Оценить объем памяти для хранения данных.
	\item Описать требования к ПО.
	\item Протестировать разработанное ПО.
	\item Сравнить реализации стеков на списке и массиве.
\end{enumerate}