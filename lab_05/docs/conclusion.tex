\Conclusion

Решение о выборе той или иной реализации очереди сильно зависит от выполняемых задач. Для представленной в данной лабораторной работе задачи (моделирование работы системы ОА) у каждой реализации есть свои преимущества и свои недостатки.

Для реализации очереди на основе кольцевого массива характерна быстрота выполняемых операций (вставки и удаления). Это решение эффективно по времени, но зачастую проигрывает по памяти, при малой заполненности очереди (менее 50\%). Главной особенностью реализации очереди на списке является удобство работы и тот факт, что максимальное количество элементов ограничено лишь объемом оперативной памяти.
