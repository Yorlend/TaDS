\Conclusion

В результате выполения лабораторной работы были исследованы подходы к хранению разреженных матриц, спроектированы и реализованы структуры данных для их хранения. Был описан алгоритм умножения разреженного вектора на разреженную матрицу и протестированно разработанное ПО.

В ходе тестирования ПО было выявлено, что использование разреженной формы представления матриц выгодно для обработки при большом количестве элементов и проценте разреженности более 30-40\%. При размерности матрицы $100\times100$ разреженный строчный формат становится неэффективным по памяти при 65-70\% заполненности, в то время как матрицы $10\times10$ неэффективны уже при 5-7\%.