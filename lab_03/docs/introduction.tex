\Introduction

Матрицы широко используются для представления информации о многих сферах деятельности. Матрицы и эффективные алгоритмы работы с ними применяются в анализе данных и машинном обучении. Часто применяют алгоритмы для работы с \textbf{разреженными} матрицами, иными словами матрицами, количество нулевых элементов в которой во много раз превышает количество ненулевых.

\textbf{Целью данной работы является} реализация алгоритмов обработки разреженных матриц, 
сравнение  эффективности  использования  этих  алгоритмов  (по  времени 
выполнения и по требуемой памяти) со стандартными алгоритмами обработки 
матриц при различном процентном заполнении матриц ненулевыми значениями 
и при различных размерах матриц. 

\textbf{Для достижения поставленной цели необходимо выполнить следующие задачи: }

\begin{itemize}[$\bullet$]
	\item исследовать подходы к хранению разреженных матриц;
	\item описать используемые структуры данных;
	\item описать алгоритм умножения разреженного вектора на разреженную матрицу;
	\item протестировать разработанное ПО;
	\item сравнить эффективность по времени и памяти умножения вектора на матрицу в разреженном и классическом представлениях.
\end{itemize}